{\actuality}
Исследования кинетических явлений (процессы переноса, оптические свойства) в квантовых системах с пониженной размерностью в настоящее время продолжают привлекать внимание, как теоретиков, так и экспериментаторов. Это связано с тем, что энергетический спектр электронов и дырок в таких системах из-за размерного квантования принципиальным образом отличается от объемного материала. Одним из наиболее перспективных в использовании и интенсивно изучаемых является эффект размерного квантования. Квантование движения частиц имеет место, когда характерный размер системы близок по величине к длине волны де Бройля носителей. Уменьшение размеров системы в одном, либо двух, трех направлениях приводит к тому, что движение частицы становится соотвественно квазидумерным, квазиодномерным или квазинульмерным. В настоящее время существует большое число видов низкоразмерных систем: квантовые ямы (КЯ), сверхрешетки (СР), квантовые проволоки (КП), квантовые доты (точки) и их системы. Именно появление размерно-квантованных зон проводимости (как и валентных зон) создает новые каналы поглощения и люминесценции, определяемые, например, переходом заряженной частицы между размерно-квантованными зонами проводимости (межподзонные переходы). При этом, что является очень важным, из-за локализации носителей сила осциллятора при таких переходах велика, что приводит к большим, порядка $10^3\,\text{cm}^{-1}$, значениям коэффициента поглощения слабой электромагнитной волны в далекой инфракрасной области спектра. Если в объемных полупроводниковых материалах электронные явления переноса определяются в основном рассеянием носителей на колебаниях кристаллической решетки, при низких температурах --- рассеянием на легированной примеси, то в размерно-ограниченных системах (квантовые ямы, гетероструктуры, квантовые проволоки и т.д.) возникает новый тип рассеяния – рассеяние носителей на шероховатой поверхности. Любое незначительное изменение размеров квантовой системы (например ширины КЯ, радиуса КП) приводит, естественно, к изменению энергии размерного квантования. Именно это изменение можно рассматривать как результат взаимодействия носителей с шероховатой поверхностью. Наиболее удачным для описания таких процессов рассеяния носителей является модель, когда размеры наноструктуры при движении заряженных частиц вдоль поверхности меняются случайным образом. В дальнейшем при описании кинетических явлений в размерно-ограниченных системах используется эта модель. Именно этот механизм рассеяния может описать большие значения подвижности носителей в области низких температур, наблюдаемые в экспериментальных исследованиях. При этом, естественно, чем меньше ширина размерно-квантованной системы, тем процессы рассеяния носителей на шероховатой поверхности становятся более активными. Поэтому механизм рассеяния носителей на шероховатой поверхности можно экспериментально выделить от других механизмов рассеяния по резкой зависимости кинетических коэффициентов от размеров наноструктуры и по влиянию поперечного электрического поля на процессы рассеяния. Эти процессы рассеяния влияют на кинетические явления по-разному в зависимости от ориентации напряженности внешнего однородного магнитного поля по отношению к оси пространственного квантования. Рассматриваемые процессы рассеяния важны в нелегированных наноструктурах в области низких температур $T$, потому что с ростом $T$ вначале важную роль начинают играть процессы рассеяния носителей на длинноволновых (акустических) колебаниях кристаллической решетки, а потом включаются процессы, связанные с взаимодействием заряженных частиц с оптическими фононами.

В настоящее время для описания кинетических явлений в размерно-ограниченных системах используется модель, в которой потенциал квантовой системы аппроксимируется параболой. Системы с квадратичным потенциалом интересны тем, что проявление эффектов размерного квантования в них происходит в достаточно больших размерно-ограниченных системах. Например, для типичных параметров параболической квантовой ямы GaAs/AlAs шаг пространственного квантования для электронов равен $14.6/a$ eV ($a$-ширина КЯ в $\AA$), т.е. при $a=10^3 \AA, \hbar\omega_e = 14.6 \text{ meV}$. Следовательно, уже при температуре $T=100 K$, должно заметно проявляться влияние размерно-квантованных уровней на кинетические свойства таких систем. Квадратичная зависимость потенциала также удобна для теоретических расчетов и позволяет получить многие характеристики полупроводниковых систем в аналитическом виде, что делает более удобным проводить детальный анализ рассматриваемых физических явлении.

В первой главе данной диссертационной работы представлен обзор экспериментов и теоретических работ, посвященных исследованию влияния процессов рассеяния носителей на шероховатой поверхности на оптические свойства и явления переноса в размерно-ограниченных системах. В ней же кратко изложена теория рассеяния носителей на шероховатой поверхности в размерно-ограниченных системах. Обсуждаются приближения, которые используются ниже в оригинальных главах диссертации.

Вторая глава диссертации посвящена расчету коэффициента поглощения света различной поляризации, позволяющий исследовать частотную зависимость поглощения света в широкой области частот и последовательно описать, как межподзонное, так и внутризонное поглощение с учетом взаимодействия электрона с шероховатой поверхностью.
В этой же главе исследуется влияние резонансного инфракрасного лазерного излучения на межзонное поглощение света в системах с пониженной размерностью (квантовые ямы, нанопроволоки), когда частота лазерного излучения равна или частоте размерного квантования или гибридной частоте.

В третьей главе диссертации проведено исследование влияния рассеяния носителей на шероховатой поверхности на процессы переноса в квантовых ямах и квантовых проволоках. Теоретические результаты сравниваются с экспериментальными данными в КЯ GaAs-AlAs, нанопроволоках Bi.

Четвертая глава диссертации посвящена исследованию явлений переноса (подвижность, термоэдс) в электрическом поле, направленном перпендикулярно оси размерно-ограниченной системы. В частности, показано, что подвижность с ростом напряженности поперечного электрического поля уменьшается, а в случае вырожденного электронного газа подвижность описывается осцилляционной кривой. Указанные особенности в подвижности возникают только при учете рассеяния носителей на шероховатой поверхности исследуемой наноструктуры.

%% {\progress} 
%% Этот раздел должен быть отдельным структурным элементом по
%% ГОСТ, но он, как правило, включается в описание актуальности
%% темы. Нужен он отдельным структурынм элемементом или нет ---
%% смотрите другие диссертации вашего совета, скорее всего не нужен.
%
%{\aim} данной работы является \ldots
%
%Для~достижения поставленной цели необходимо было решить следующие {\tasks}:
%\begin{enumerate}
%  \item Исследовать, разработать, вычислить и~т.\:д. и~т.\:п.
%  \item Исследовать, разработать, вычислить и~т.\:д. и~т.\:п.
%  \item Исследовать, разработать, вычислить и~т.\:д. и~т.\:п.
%  \item Исследовать, разработать, вычислить и~т.\:д. и~т.\:п.
%\end{enumerate}
%
%
%{\novelty}
%\begin{enumerate}
%  \item Впервые \ldots
%  \item Впервые \ldots
%  \item Было выполнено оригинальное исследование \ldots
%\end{enumerate}
%
%{\influence} \ldots
%
%{\methods} \ldots
%
%{\defpositions}
%\begin{enumerate}
%  \item Первое положение
%  \item Второе положение
%  \item Третье положение
%  \item Четвертое положение
%\end{enumerate}
%
%{\reliability} полученных результатов обеспечивается \ldots \ Результаты находятся в соответствии с результатами, полученными другими авторами.


{\probation}
Основные результаты работы докладывались~на:
перечисление основных конференций, симпозиумов и~т.\:п.

%{\contribution} Автор принимал активное участие в обсуждении и постановке задач, рассмотренных в диссертационной работе. Все основные оригинальные расчеты проведены автором самостоятельно.

%\publications\ Основные результаты по теме диссертации изложены в ХХ печатных изданиях~\cite{Sineavsky2006,Karapetyan2006,Karapetyan2011,Karapetyan2012,Solovenko2012,Karapetyan2014,Karapetyan2017},
%Х из которых изданы в журналах, рекомендованных ВАК~\cite{Sineavsky2006,Karapetyan2006,Karapetyan2011,Karapetyan2012,Solovenko2012,Karapetyan2014,Karapetyan2017}, 
%ХХ --- в тезисах докладов~\cite{Lermontov,Management}.

\ifthenelse{\equal{\thebibliosel}{0}}{% Встроенная реализация с загрузкой файла через движок bibtex8
    \publications\ Основные результаты по теме диссертации изложены в XX печатных изданиях, 
    X из которых изданы в журналах, рекомендованных ВАК, 
    X "--- в тезисах докладов.%
}{% Реализация пакетом biblatex через движок biber
%Сделана отдельная секция, чтобы не отображались в списке цитированных материалов
    \begin{refsection}%       
        \printbibliography[heading=countauthorvak, env=countauthorvak, keyword=biblioauthorvak, section=1]%
        \printbibliography[heading=countauthornotvak, env=countauthornotvak, keyword=biblioauthornotvak, section=1]%
        \printbibliography[heading=countauthorconf, env=countauthorconf, keyword=biblioauthorconf, section=1]%
        \printbibliography[heading=countauthor, env=countauthor, keyword=biblioauthor, section=1]%
        \publications\ Основные результаты по теме диссертации изложены
       в \arabic{citeauthor} печатных изданиях,
       \arabic{citeauthorvak}~из которых изданы в журналах,
       рекомендованных ВАК, \arabic{citeauthorconf}~в тезисах
       конференций.\nocite{Karapetyan2006,Sineavsky2006,Karapetyan2011a,Karapetyan2012a,Solovenko2012,Karapetyan2014,Karapetyan2017,Minsk2008a,PGU2006,PGU2006,PGU2008,PGU2010,PGU2012,Ul2004,Ul2005,Tir2005,Orlando2005,CFM-2005,Ul2006,Ul2006,Saransk2006,Tir2007,Chisinau2007,Ul2008,Minsk2008,Ul2009,Tir2009,Chisinau2009,Kaluga2009,CFM-2009,Nanopiter-2010,Ul2010,Ul2011,Tir2011,Ul2012,Ul2013,Tir2013,Ul2015,Tir2015}.
    \end{refsection}
}
%При использовании пакета \verb!biblatex! для автоматического подсчёта
%количества публикаций автора по теме диссертации, необходимо
%их здесь перечислить с использованием команды \verb!\nocite!.
    

