
{\actuality} Исследования кинетических явлений (процессы переноса, оптические свойства) в квантовых системах с пониженной размерностью в настоящее время продолжают привлекать внимание как теоретиков так и экспериментаторов. Это связано с тем, что энергетический спектр электронов и дырок в таких системах из-за размерного квантования принципиальным образом отличается от объемного материала. Именно появление размерно-квантованных зон проводимости (как и валентных зон) создает новые каналы поглощения и люминесценции, определяемые, например, с переходом заряженной частицы между размерно-квантованными зонами проводимости (межподзонные переходы). При этом, что является очень важным, из-за локализации носителей сила осциллятора при таких переходах велика, что приводит к большим, порядка $10^3\,\text{cm}^{-1}$, значениям коэффициента поглощения слабой электромагнитной волны в далекой инфракрасной области спектра. Если в объемных полупроводниковых материалах электронные явления переноса определяются в основном рассеянием носителей на колебаниях кристаллической решетки, при низких температурах рассеянием на легированной примеси, то в размерно-ограниченных системах (квантовые ямы, гетероструктуры, квантовые проволоки и т.д.) возникает новый тип рассеяния – рассеяние носителей на шероховатой поверхности. Именно этот механизм рассеяния может определять величину электропроводности в области низких температур. При этом, естественно, чем меньше ширина размерно-квантованной системы, тем процессы рассеяния носителей на шероховатой поверхности становятся наиболее активными. Эти процессы влияют на кинетические явления по разному в зависимости от ориентации напряженности внешнего однородного магнитного поля по отношению к оси пространственного квантования. Рассматриваемые процессы рассеяния важны в области низких температур $T$, потому что с ростом $T$ вначале важную роль начинают играть процессы рассеяния носителей на длинноволновых (акустических) колебаниях кристаллической решетки, а потом включаются процессы, связанные с взаимодействием заряженных частиц с оптическими фононами.

% {\progress} 
% Этот раздел должен быть отдельным структурным элементом по
% ГОСТ, но он, как правило, включается в описание актуальности
% темы. Нужен он отдельным структурынм элемементом или нет ---
% смотрите другие диссертации вашего совета, скорее всего не нужен.

{\aim} данной работы является \ldots

Для~достижения поставленной цели необходимо было решить следующие {\tasks}:
\begin{enumerate}
  \item Исследовать, разработать, вычислить и~т.\:д. и~т.\:п.
  \item Исследовать, разработать, вычислить и~т.\:д. и~т.\:п.
  \item Исследовать, разработать, вычислить и~т.\:д. и~т.\:п.
  \item Исследовать, разработать, вычислить и~т.\:д. и~т.\:п.
\end{enumerate}


{\novelty}
\begin{enumerate}
  \item Впервые \ldots
  \item Впервые \ldots
  \item Было выполнено оригинальное исследование \ldots
\end{enumerate}

{\influence} \ldots

{\methods} \ldots

{\defpositions}
\begin{enumerate}
  \item Первое положение
  \item Второе положение
  \item Третье положение
  \item Четвертое положение
\end{enumerate}

{\reliability} полученных результатов обеспечивается \ldots \ Результаты находятся в соответствии с результатами, полученными другими авторами.


{\probation}
Основные результаты работы докладывались~на:
перечисление основных конференций, симпозиумов и~т.\:п.

{\contribution} Автор принимал активное участие \ldots

%\publications\ Основные результаты по теме диссертации изложены в ХХ печатных изданиях~\cite{Sokolov,Gaidaenko,Lermontov,Management},
%Х из которых изданы в журналах, рекомендованных ВАК~\cite{Sokolov,Gaidaenko}, 
%ХХ --- в тезисах докладов~\cite{Lermontov,Management}.

\ifthenelse{\equal{\thebibliosel}{0}}{% Встроенная реализация с загрузкой файла через движок bibtex8
    \publications\ Основные результаты по теме диссертации изложены в XX печатных изданиях, 
    X из которых изданы в журналах, рекомендованных ВАК, 
    X "--- в тезисах докладов.%
}{% Реализация пакетом biblatex через движок biber
%Сделана отдельная секция, чтобы не отображались в списке цитированных материалов
    \begin{refsection}%
        \printbibliography[heading=countauthornotvak, env=countauthornotvak, keyword=biblioauthornotvak, section=1]%
        \printbibliography[heading=countauthorvak, env=countauthorvak, keyword=biblioauthorvak, section=1]%
        \printbibliography[heading=countauthorconf, env=countauthorconf, keyword=biblioauthorconf, section=1]%
        \printbibliography[heading=countauthor, env=countauthor, keyword=biblioauthor, section=1]%
        \publications\ Основные результаты по теме диссертации изложены в \arabic{citeauthor} печатных изданиях\nocite{bib1,bib2}, 
        \arabic{citeauthorvak} из которых изданы в журналах, рекомендованных ВАК\nocite{vakbib1,vakbib2}, 
        \arabic{citeauthorconf} "--- в тезисах докладов\nocite{confbib1,confbib2}.
    \end{refsection}
}
При использовании пакета \verb!biblatex! для автоматического подсчёта
количества публикаций автора по теме диссертации, необходимо
их здесь перечислить с использованием команды \verb!\nocite!.
    

