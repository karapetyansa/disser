{\actuality}
\ifsynopsis
Исследования кинетических явлений (процессы переноса, оптические свойства) в наносистемах продолжают привлекать внимание как теоретиков, так и экспериментаторов. Это связано с тем, что энергетический спектр носителей в таких системах (движение электронов вдоль ширин квантовой ямы, вдоль радиуса квантовой проволоки являются финитными) принципиальным образом отличаются от объемных материалов.
Именно появление размерно-квантованных зон проводимости (как и валентных зон) создает новые каналы поглощения и люминесценции света, что безусловно расширяет возможности приборного приложения таких квантовых систем в оптоэлектронике. Изучение процессов рассеяния носителей на шероховатой поверхности является важным и актуальным, так как позволяет понять большие значения подвижности, экспериментально наблюдаемые в тонких наноструктурах, объяснить большие (порядка $10^3$см$^{-1}$) значения коэффициента поглощения слабой электромагнитной волны в инфракрасной области спектра. При низких температурах и тонких нелегированных наносистемах рассеяние носителей на шероховатой поверхности может оказаться доминирующим механизмом рассеяния, определяющим физические свойства квантовых систем. Заметное влияние внешних полей (например поперечного электрического поля) на процессы рассеяния носителей на шероховатой поверхности представляются перспективным для создания различных приборов из наноструктур, кинетическими процессами в которых можно управлять.
Таким образом, исследования кинетических явлений в размерно-ограниченных системах являются перспективными и актуальными.
\else
Исследования кинетических явлений (процессы переноса, оптические свойства) в квантовых системах с пониженной размерностью в настоящее время продолжают привлекать внимание, как теоретиков, так и экспериментаторов. Это связано с тем, что энергетический спектр электронов и дырок в таких системах из-за размерного квантования принципиальным образом отличается от объемного материала.

Одним из наиболее перспективных в использовании и интенсивно изучаемых является эффект размерного квантования. Квантование движения частиц имеет место, когда характерный размер системы близок по величине к длине волны де Бройля носителей. Уменьшение размеров системы в одном, либо двух, трех направлениях приводит к тому, что движение частицы становится соотвественно квазидумерным, квазиодномерным или квазинульмерным. В настоящее время существует большое число видов низкоразмерных систем: квантовые ямы (КЯ), сверхрешетки (СР), квантовые проволоки (КП), квантовые доты (точки) и их системы. Именно появление размерно-квантованных зон проводимости (как и валентных зон) создает новые каналы поглощения и люминесценции, определяемые, например, переходом заряженной частицы между размерно-квантованными зонами проводимости (межподзонные переходы). При этом, что является очень важным, из-за локализации носителей сила осциллятора при таких переходах велика, что приводит к большим, порядка $10^3\,\text{cm}^{-1}$, значениям коэффициента поглощения слабой электромагнитной волны в далекой инфракрасной области спектра.

Если в объемных полупроводниковых материалах электронные явления переноса определяются в основном рассеянием носителей на колебаниях кристаллической решетки, при низких температурах --- рассеянием на легированной примеси, то в размерно-ограниченных системах (квантовые ямы, гетероструктуры, квантовые проволоки и т.д.) возникает новый тип рассеяния – рассеяние носителей на шероховатой поверхности. Любое незначительное изменение размеров квантовой системы (например ширины КЯ, радиуса КП) приводит, естественно, к изменению энергии размерного квантования. Именно это изменение можно рассматривать как результат взаимодействия носителей с шероховатой поверхностью. Наиболее удачным для описания таких процессов рассеяния носителей является модель, когда размеры наноструктуры при движении заряженных частиц вдоль поверхности меняются случайным образом. В дальнейшем при описании кинетических явлений в размерно-ограниченных системах используется эта модель. Именно этот механизм рассеяния может описать большие значения подвижности носителей в области низких температур, наблюдаемые в экспериментальных исследованиях. При этом, естественно, чем меньше ширина размерно-квантованной системы, тем процессы рассеяния носителей на шероховатой поверхности становятся более активными. Поэтому механизм рассеяния носителей на шероховатой поверхности можно экспериментально выделить от других механизмов рассеяния по резкой зависимости кинетических коэффициентов от размеров наноструктуры и по влиянию поперечного электрического поля на процессы рассеяния. Эти процессы рассеяния влияют на кинетические явления по-разному в зависимости от ориентации напряженности внешнего однородного магнитного поля по отношению к оси пространственного квантования. Рассматриваемые процессы рассеяния важны в нелегированных наноструктурах в области низких температур $T$, потому что с ростом $T$ вначале важную роль начинают играть процессы рассеяния носителей на длинноволновых (акустических) колебаниях кристаллической решетки, а потом включаются процессы, связанные с взаимодействием заряженных частиц с оптическими фононами.

В настоящее время для описания кинетических явлений в размерно-ограниченных системах используется модель, в которой потенциал квантовой системы аппроксимируется параболой. Системы с квадратичным потенциалом интересны тем, что проявление эффектов размерного квантования в них происходит в достаточно больших размерно-ограниченных системах. Например, для типичных параметров параболической квантовой ямы GaAs/AlAs шаг пространственного квантования для электронов равен $14.6/a$ eV ($a$-ширина КЯ в $\AA$), т.е. при $a=10^3 \AA, \hbar\omega_e = 14.6 \text{ meV}$. Следовательно, уже при температуре $T=100 K$, должно заметно проявляться влияние размерно-квантованных уровней на кинетические свойства таких систем. Квадратичная зависимость потенциала также удобна для теоретических расчетов и позволяет получить многие характеристики полупроводниковых систем в аналитическом виде, что делает более удобным проводить детальный анализ рассматриваемых физических явлении.
\fi

\underline{\textbf{Связь работы с научными программами, планами, темами}}
Диссертационная работа выполнялась по планам кафедры теоретической физики и научно исследовательской лаборатории “Полярон” Приднестровского государственного университета им Т.Г. Шевченко, а так же по научному плану лаборатории физической кинетики ИПФ АН Молдовы. Некоторые разделы работы выполнены в рамках Украинского научно-технического центра (грант №5062), STCU (грант №5929)

\underline{\textbf{Объект исследования:}} Полупроводниковые и полуметаллические низкоразмерные структуры (квантовые ямы и квантовые проволоки) во внешних постоянных электрическом и магнитном полях.

\underline{\textbf{Предмет исследования:}} Электропроводность, термоэдс, коэффициент поглощения света в низкоразмерных структурах во внешних полях с учетом взаимодействия носителей заряда с шероховатой поверхностью.

\underline{\textbf{Цель работы заключается}}
в теоретическом исследовании влияния процессов рассеяния носителей на шероховатой поверхности наноструктуры во внешних полях (поперечных электрическом и магнитном полях) на оптические свойства и процессы переноса заряда (подвижность, термоэдс) в размерно-ограниченных системах (квантовые ямы, нанопроволоки). Для достижежния цели следует решить \textbf{следующие задачи}
\begin{itemize}
	\item Теоретически исследованы процессы рассеяния носителей на шероховатой поверхности в размерно-ограниченных системах (квантовые ямы (КЯ), нанопроволоки (КП)) в модели параболического потенциала во внешних электрическом и магнитном полях.
	\item Подробно изучено влияние поперечных электрического и магнитного полей на процессы переноса заряда (подвижность, термоэдс) в наноструктурах с учетом рассеяния носителей на шероховатой поверхности.
	\item Исследовано влияние ИК лазерного излучения на межзонное поглощение слабой электромагнитной волны в нанопровлоках, когда частота излучения равна частоте размерного квантования (размерно-индуцированный резонанс) или гибридной частоте (магнито-инфракрасный резонанс).
	\item Выяснена роль процессов рассеяния на шероховатой поверхности на внутризонные, межзонные и межподзонные оптические переходы в размерно-ограниченных системах.
\end{itemize}

\underline{\textbf{Научная новизна}}
\begin{enumerate}
	\item Впервые получены подвижность и электропроводность в размерно-ограниченных системах (прямоугольные, параболические квантовые ямы, нанопроволоки) с одновременным учетом рассеяния носителей на шероховатой поверхности и упругого рассеяния на длинноволновых (акустических) колебаниях без использования классического уравнения Больцмана из общих соотношений неравновесной квантовой статистики в приближении времени релаксации вычисляется электропроводность.
	\item Впервые подробно исследовано влияние однородного магнитного поля $\vect{H}$ на электропроводность в размерно-квантованных системах (квантовые ямы, параболические нанопроволоки) с учетом рассеяния носителей на шероховатой поверхности.
	\item Впервые исследовано влияние интенсивного лазерного излучения на межзонное поглощение слабой электромагнитной волны в квантовых проволоках. 
	\item Впервые проведены теоретические исследования электропроводности, термоэдс в размерно-квантованных системах в присутствии однородного электрического поля $\vect{E}$, направленного вдоль оси пространственного квантования.
\end{enumerate}

\underline{\textbf{Теоретическая значимость}} 1) предложен метод расчета кинетических коэффициентов (электропроводность, термоэдс), позволяющих описать явления переноса в размерно-квантованных системах без использования классического уравнения Больцмана;  2) Получены выражения для времени релаксации при учете рассеяния носителей заряда на шероховатой поверхности во внешних электрическом и магнитном полях, которое может быть использована для вычисления кинетических коэффициентов и коэффициентов поглощения и люминесценции. 3) предложен метод управления кинетическими процессами в квантовых системах с пониженной размерностью в электрическом поле, направленном параллельно оси размерного квантования. 

\underline{\textbf{Практическая значимость}} работы определяется возможностью управления процессами переноса и оптическими свойствами наносистем внешними полями. Результаты исследования могут быть использованы для улучшения характеристик существующих опто-электронных приборов, а также создания принципиально новых опто-электронных приборов.

\underline{\textbf{Достоверность полученных результатов}} подтверждается с использованием апробированных методов математического анализа (метод случайных функций. алгебра Бозе-операторов, метод кумулянт), использованием известных положений фундаментальных наук (например, квантовой статистики)

{\defpositions}
\begin{enumerate}
	\item Результаты теоретических исследований влияния внешних электрического и магнитного полей на процессы рассеяния носителей на шероховатой поверхности в наноструктурах (квантовые ямы, квантовые проволоки) с параболическим потенциалом. Сформулированы условия на температуру и размеры квантовой системы, когда процессы рассеяния на шероховатой поверхности являются доминирующими в нелегированных размерно-ограниченных системах.
	\item Влияние электрического поля, направленного перпендикулярно поверхности размерно-ограниченной квантовой системы и магнитного поля на процессы переноса заряда (подвижность, термоэдс) при учете процессов рассеяния носителей на шероховатой поверхности.
	\item Исследования по межзонному поглощению слабой электромагнитной волны в параболических квантовых проволоках в поле ИК лазерного излучения, частота которого равна или частоте размерного квантования (размерно-инфракрасный резонанс) или гибридной частоте (магнито-инфракрасный резонанс).
	\item Исследования особенностей коэффициента поглощения света различной поляризации при межподзонных, внутризонных переходах с учетом взаимодействия носителей с шероховатой поверхностью в квантовых ямах в широкой области частот.
\end{enumerate}

\ifsynopsis
\else
В первой главе данной диссертационной работы представлен обзор экспериментов и теоретических работ, посвященных исследованию влияния процессов рассеяния носителей на шероховатой поверхности на оптические свойства и явления переноса в размерно-ограниченных системах. В ней же кратко изложена теория рассеяния носителей на шероховатой поверхности в размерно-ограниченных системах. Обсуждаются приближения, которые используются ниже в оригинальных главах диссертации.

Вторая глава диссертации посвящена расчету коэффициента поглощения света различной поляризации, позволяющий исследовать частотную зависимость поглощения света в широкой области частот и последовательно описать, как межподзонное, так и внутризонное поглощение с учетом взаимодействия электрона с шероховатой поверхностью.
В этой же главе исследуется влияние резонансного инфракрасного лазерного излучения на межзонное поглощение света в системах с пониженной размерностью (квантовые ямы, нанопроволоки), когда частота лазерного излучения равна или частоте размерного квантования или гибридной частоте.

В третьей главе диссертации проведено исследование влияния рассеяния носителей на шероховатой поверхности на процессы переноса в квантовых ямах и квантовых проволоках. Теоретические результаты сравниваются с экспериментальными данными в КЯ GaAs-AlAs, нанопроволоках Bi.

Четвертая глава диссертации посвящена исследованию явлений переноса (подвижность, термоэдс) в электрическом поле, направленном перпендикулярно оси размерно-ограниченной системы. В частности, показано, что подвижность с ростом напряженности поперечного электрического поля уменьшается, а в случае вырожденного электронного газа подвижность описывается осцилляционной кривой. Указанные особенности в подвижности возникают только при учете рассеяния носителей на шероховатой поверхности исследуемой наноструктуры.
\fi

%% {\progress} 
%% Этот раздел должен быть отдельным структурным элементом по
%% ГОСТ, но он, как правило, включается в описание актуальности
%% темы. Нужен он отдельным структурынм элемементом или нет ---
%% смотрите другие диссертации вашего совета, скорее всего не нужен.
%
%{\aim} данной работы является \ldots
%
%Для~достижения поставленной цели необходимо было решить следующие {\tasks}:
%\begin{enumerate}
%  \item Исследовать, разработать, вычислить и~т.\:д. и~т.\:п.
%  \item Исследовать, разработать, вычислить и~т.\:д. и~т.\:п.
%  \item Исследовать, разработать, вычислить и~т.\:д. и~т.\:п.
%  \item Исследовать, разработать, вычислить и~т.\:д. и~т.\:п.
%\end{enumerate}
%
%
%{\novelty}
%\begin{enumerate}
%  \item Впервые \ldots
%  \item Впервые \ldots
%  \item Было выполнено оригинальное исследование \ldots
%\end{enumerate}
%
%{\influence} \ldots
%
%{\methods} \ldots
%
%{\defpositions}
%\begin{enumerate}
%  \item Первое положение
%  \item Второе положение
%  \item Третье положение
%  \item Четвертое положение
%\end{enumerate}
%
%{\reliability} полученных результатов обеспечивается \ldots \ Результаты находятся в соответствии с результатами, полученными другими авторами.


{\probation}
Основные результаты работы докладывались~на международных конференциях «Опто-, наноэлектроника, нанотехнологии и микросистемы» (Ульяновск 2004, 2005, 2006, 2008, 2009, 2010, 2011, 2012, 2013, 2015), «Conferinta Fizicienilor din Moldova» (Кишинев, Молдова 2005, 2009), «24th International Conference on Low Temperature Physics» (Орладно, США 2005), «Математическое моделирование в образовании, науке и производстве» (Тирасполь, Молдова 2005, 2007, 2009, 2011, 2013, 2015, 2017), «Материалы нано-, микро-, оптоэлектроники и волоконной оптики: физические свойства и применение» (Саранск 2006), «Physics of low-dimensional structures» (Кишинев, Молдова 2007), «Квантовая электроника» (Минск, Беларусь 2008), «6th International Conference on Microelectronics and Computer» (Кишинев, Молдова 2009), «Вторая Всероссийская Школа-семинар студентов, аспирантов и молодых ученых по направлению «Наноинженерия» (Саранск 2009), «Foundamentals of Elctronic Nanosystems NANOPITER-2010» (Санкт-Петербург 2010), «X Международная конференция молодых ученых и специалистов «Оптика – 2017» (Санкт-Петербург 2017), VI-я Международная конференция "Телекоммуникации, электроника и информатика" (Кишинев 2018), The 9th International Conference on Materials Science and Condensed Matter Physics (Кишинев 2018), 2018 IEEE 8-TH INTERNATIONAL CONFERENCE ON NANOMATERIALS: APPLICATIONS AND PROPERTIES (2018)

%{\contribution} Автор принимал активное участие в обсуждении и постановке задач, рассмотренных в диссертационной работе. Все основные оригинальные расчеты проведены автором самостоятельно.

%\publications\ Основные результаты по теме диссертации изложены в ХХ печатных изданиях~\cite{Sineavsky2006,Karapetyan2006,Karapetyan2011,Karapetyan2012,Solovenko2012,Karapetyan2014,Karapetyan2017},
%Х из которых изданы в журналах, рекомендованных ВАК~\cite{Sineavsky2006,Karapetyan2006,Karapetyan2011,Karapetyan2012,Solovenko2012,Karapetyan2014,Karapetyan2017}, 
%ХХ --- в тезисах докладов~\cite{Lermontov,Management}.

\ifthenelse{\equal{\thebibliosel}{0}}{% Встроенная реализация с загрузкой файла через движок bibtex8
    \publications\ Основные результаты по теме диссертации изложены в XX печатных изданиях, 
    X из которых изданы в журналах, рекомендованных ВАК, 
    X "--- в тезисах докладов.%
}{% Реализация пакетом biblatex через движок biber
%Сделана отдельная секция, чтобы не отображались в списке цитированных материалов
    \begin{refsection}%       
        \printbibliography[heading=countauthorvak, env=countauthorvak, keyword=biblioauthorvak, section=1]%
        \printbibliography[heading=countauthornotvak, env=countauthornotvak, keyword=biblioauthornotvak, section=1]%
        \printbibliography[heading=countauthorconf, env=countauthorconf, keyword=biblioauthorconf, section=1]%
        \printbibliography[heading=countauthor, env=countauthor, keyword=biblioauthor, section=1]%
        \publications\ Основные результаты по теме диссертации изложены
       в \arabic{citeauthor} печатных изданиях,
       \arabic{citeauthorvak}~из которых изданы в журналах,
       рекомендованных ВАК, \arabic{citeauthorconf}~в тезисах и трудах международных конференций\nocite{Karapetyan2006,Sineavsky2006,Karapetyan2011a,Karapetyan2012a,Solovenko2012,Karapetyan2014,Karapetyan2017,Minsk2008a,PGU2006,PGU2008,PGU2010,PGU2012,PGU2018-1,PGU2018-2,Ul2004,Ul2005,Tir2005,Orlando2005,CFM-2005,Ul2006,Ul2006,Saransk2006,Tir2007,Chisinau2007,Ul2008,Minsk2008,Ul2009,Tir2009,Chisinau2009,Kaluga2009,CFM-2009,Nanopiter-2010,Ul2010,Ul2011,Tir2011,Ul2012,Ul2013,Tir2013,Ul2015,Tir2015,Optica2017,Tiraspol2017,MSCMP2018,Nap-2018-1,Nap-2018-2,Ictei2018}.
    \end{refsection}
}
%При использовании пакета \verb!biblatex! для автоматического подсчёта
%количества публикаций автора по теме диссертации, необходимо
%их здесь перечислить с использованием команды \verb!\nocite!.
    

