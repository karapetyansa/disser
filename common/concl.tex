%% Согласно ГОСТ Р 7.0.11-2011:
%% 5.3.3 В заключении диссертации излагают итоги выполненного исследования, рекомендации, перспективы дальнейшей разработки темы.
%% 9.2.3 В заключении автореферата диссертации излагают итоги данного исследования, рекомендации и перспективы дальнейшей разработки темы.
\begin{enumerate}
  \item Из общих соотношений неравновесной квантовой статистики в приближении времени релаксации вычисляется электропроводность в размерно-ограниченных системах (прямоугольные, параболические квантовые ямы, нанопроволоки) с одновременным учетом рассеяния носителей на шероховатой поверхности и упругого рассеяния на длинноволновых (акустических) колебаниях. Полученные теоретические результаты по величине подвижности, по зависимости подвижности от температуры,  от размеров наноструктуры находят экспериментальное подтверждение в разнообразных квантовых системах. Именно из сравнения теоретических результатов с экспериментальными данными позволяет провести оценки параметров флуктуирующей поверхности наноструктуры. Сформулированны условия на размеры квантовых наносистем, температуру, когда упругие процессы рассеяния электронов на шероховатой поверхности становятся доминирующими по сравнению с рассеянием носителей на акустических фононах.
  \item Подробно исследовано влияние однородного магнитного поля $\vect{H}$ на электропроводность в размерно квантованных системах (квантовые ямы, параболические нанопроволоки) с учетом рассеяния носителей на шероховатой поверхности. Показано, что с ростом напряженности продольного магнитного поля подвижность увеличивается, что связано с уменьшением радиуса локализации носителей тока в исследуемых наноструктурах, т.е. к уменьшению вероятности рассеяния носителей на шероховатой поверхности. В поперечном магнитном поле подвижность с ростом напряженности поля уменьшается. Такое поведение подвижности от $\vect{H}$ связано с тем, что в скрещенных электрическом и магнитном полях носители с дрейфовой скоростью перемещаются вдоль оси пространственного квантования, поэтому активно участвуют в процессах рассеяния на шероховатой поверхности исследуемой наноструктуры. Некоторые теоретические результаты сравниваются с экспериментальными данными для электропроводности в магнитном поле для нанопроволок Bi
  \item Впервые исследовано влияние интенсивного лазерного излучения на межзонное поглощение слабой электромагнитной волны в квантовых проволоках. Показано, что когда частота лазерного излучения равна или частоте размерного квантования (размерно-инфракрасный резонанс) или, в присутствии поперечного магнитного поля, гибридной частоте (магнито-инфракрасный резонанс) то лазерная подсветка определяет форму осцилляций коэффициента поглощения света. В частности показано, что второй пик магнетопоглощения расщепляется на два пика, полуширина которых и расстояние между которыми зависят от интенсивности резонансного лазерного излучения. Проведено подробное исследование динамики изменения частотной зависимости коэффициента межзонного поглощения света при увеличении интенсивности резонансного лазерного излучения. Именно существенное изменение частотной зависимости коэффициента межзонного поглощения света в поле резонансного лазерного излучения дает возможность наблюдать заметное поглощение света при частотах, когда в отсутствии лазерной подсветки исследуемая квантовая система не поглощает электромагнитную волну.
  Важно отметить, что заметное влияние лазерного излучения на оптические характеристики размерно-ограниченных систем осуществляется при небольших (экспериментально достигаемых) интенсивностях ИК излучения, что позволяет надеяться на экспериментальное обнаружение предсказанного эффекта. Именно в квантованных нанопроволоках, из-за одномерного движения носителей, возникают особенности в плотности электронных состояний, что приводит к наиболее яркому проявлению влияния резонансного лазерного излучения на оптические характеристики исследуемой квантовой системы.
  \item Впервые проведены теоретические исследования электропроводности, термоэдс в размерно0квантованных системах (параболические квантовые ямы, параболические нанопроволоки) в присутствии однородного электрического поля $\vect{E}$, направленного вдоль оси пространственного квантования. 
  Показано, что только при учете взаимодействия носителей с шероховатой поверхностью, подвижность, термоэдс с ростом $\vect{E}$ уменьшаются. В случае вырожденного электронного (дырочного) газа подвижность, термоэдс осциллируют при изменнии поперечного однородного электрического поля. Предложена физическая интерпретация такого поведения кинетических коэффициентов от  $\vect{E}$. Проведены детальные исследования влияния однородного магнитного поля различной ориентации по отношению к $\vect{E}$ на подвижность квантовых проволок Bi.
\end{enumerate}
