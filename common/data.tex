%%% Основные сведения %%%
\newcommand{\thesisAuthor}             % Диссертация, ФИО автора
{%
    \texorpdfstring{% \texorpdfstring takes two arguments and uses the first for (La)TeX and the second for pdf
        Карапетян Сергей Артурович% так будет отображаться на титульном листе или в тексте, где будет использоваться переменная
    }{%
        Карапетян Сергей Артурович% эта запись для свойств pdf-файла. В таком виде, если pdf будет обработан программами для сбора библиографических сведений, будет правильно представлена фамилия.
    }%
}
\newcommand{\thesisUdk}                % Диссертация, УДК
{538.935, 538.958, 538.971, 538.975, 535.341}
% 538.935 Электронные Явления переноса (кроме переноса в квантовых жидкостях и твердых телах)
% 538.971 Физика поверхностей и границ раздела (включая эмиссию и столкновение)
% 538.975 Физика тонких пленок, нитевидных кристаллов и дендритов
% 535.341 Коэффициент поглощения. Коэффициент экстинкции и аналогичные характеристики поглощения
\newcommand{\thesisTitle}              % Диссертация, название
{\texorpdfstring{\MakeUppercase{Влияние рассеяния носителей на шероховатой поверхности на кинетические и оптические процессы в размерно-квантованных системах}}{Влияние шероховатой поверхности на кинетические явления во внешних полях}}
\newcommand{\thesisSpecialtyNumber}    % Диссертация, специальность, номер
{\texorpdfstring{01.04.10}{01.04.10}}
\newcommand{\thesisSpecialtyTitle}     % Диссертация, специальность, название
{\texorpdfstring{Физика полупроводников}{Физика полупроводников}}
\newcommand{\thesisDegree}             % Диссертация, научная степень
{кандидата физико-математических наук}
\newcommand{\thesisDegreeShort}        % Диссертация, ученая степень, краткая запись
{\todo{канд. физ.-мат. наук}}
\newcommand{\thesisCity}               % Диссертация, город защиты
{Тирасполь}
\newcommand{\thesisYear}               % Диссертация, год защиты
{2019}
\newcommand{\thesisOrganization}       % Диссертация, организация
{Государственное образовательное учреждение высшего образования Приднестровский госудаврственный университет им.~Т.Г.~Шевченко}

\newcommand{\thesisInOrganization}       % Диссертация, организация в предложном падеже: Работа выполнена в ...
{Приднестровском Государственном Университете им.~Т.Г.~Шевченко}

\newcommand{\supervisorFio}            % Научный руководитель, ФИО
{Синявский Элерланж Петрович}
\newcommand{\supervisorRegalia}        % Научный руководитель, регалии
{д-р~физ.-мат. наук, проф.}

\newcommand{\opponentOneFio}           % Оппонент 1, ФИО
{\todo{Фамилия Имя Отчество}}
\newcommand{\opponentOneRegalia}       % Оппонент 1, регалии
{\todo{доктор физико-математических наук, профессор}}
\newcommand{\opponentOneJobPlace}      % Оппонент 1, место работы
{\todo{для места работы}}
\newcommand{\opponentOneJobPost}       % Оппонент 1, должность
{\todo{должность}}

\newcommand{\opponentTwoFio}           % Оппонент 2, ФИО
{\todo{Фамилия Имя Отчество}}
\newcommand{\opponentTwoRegalia}       % Оппонент 2, регалии
{\todo{доктор физико-математических наук}}
\newcommand{\opponentTwoJobPlace}      % Оппонент 2, место работы
{\todo{Основное место работы}}
\newcommand{\opponentTwoJobPost}       % Оппонент 2, должность
{\todo{должность}}

\newcommand{\leadingOrganizationTitle} % Ведущая организация, дополнительные строки
{\todo{Ведущая организация}}

\newcommand{\defenseDate}              % Защита, дата
{\todo{DD mmmmmmmm YYYY~г.~в~XX часов}}
\newcommand{\defenseCouncilNumber}     % Защита, номер диссертационного совета
{\todo{NN}}
\newcommand{\defenseCouncilTitle}      % Защита, учреждение диссертационного совета
{\todo{Название учреждения}}
\newcommand{\defenseCouncilAddress}    % Защита, адрес учреждение диссертационного совета
{\todo{Адрес}}

\newcommand{\defenseSecretaryFio}      % Секретарь диссертационного совета, ФИО
{\todo{Фамилия Имя Отчество}}
\newcommand{\defenseSecretaryRegalia}  % Секретарь диссертационного совета, регалии
{\todo{д-р~физ.-мат. наук}}            % Для сокращений есть ГОСТы, например: ГОСТ Р 7.0.12-2011 + http://base.garant.ru/179724/#block_30000

\newcommand{\synopsisLibrary}          % Автореферат, название библиотеки
{\todo{Название библиотеки}}
\newcommand{\synopsisDate}             % Автореферат, дата рассылки
{\todo{DD mmmmmmmm YYYY года}}

\newcommand{\keywords}%                 % Ключевые слова для метаданных PDF диссертации и автореферата
{}