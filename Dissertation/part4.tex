\chapter{Подвижность в нанопроволоках в поперечных электрическом поле с учетом рассеяния на шероховатой поверхности} \label{chapt4}

\section{Исследования подвижности носителей в квантовых ямах в постоянном поперечном электрическом поле} \label{sect4_1}

В параболических квантовых ямах (ПКЯ), когда постоянное электрическое поле ...

\section{Влияние поперечного электрического поля на подвижность в нанопроволоках} \label{sect4_2}

Для квантовых проволок электрическое поле...

\section{Особенности подвижности в нанопроволоках в поперечных электрическом и магнитном полях} \label{sect4_3}

В присутствии однородного квантующего магнитного поля энергетический спектр носителей в квантовых проволоках заметным образом меняется. В модели параболического потенциала для нанопроволок радиуса...

\section{Термоэдс в нанопроволоках Bi в попереченом постоянном электрическом поле}\label{sect4_4}

В квантовых проволоках, как следствие одномерности исследуемой наносистемы, на дне размерно-квантованных зон возникают особенности в плотности энергетических состояний. Именно это обстоятельство приводит, в частности, к особенностям оптических свойств нанопроволок...
