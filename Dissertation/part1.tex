\chapter{Рассеяния носителей на шероховатой поверхности} \label{chapt1}

\section{Обзор литературы} \label{sect1_1}

\section{Теория рассеяния носителей на шероховатой поверхности в размерно-ограниченных системах} \label{sect1_2}
Из-за неровности поверхности случайным образом меняется ширина a размерно-ограниченной системы, что приводит к флуктуации энергии размерного квантования $E_n$ при движении носителя параллельно поверхности исследуемой квантовой системы. Следовательно, энергия взаимодействия электрона (дырки) с шероховатой поверхностью в случае двухмерного электронного газа может быть записана в следующем виде \cite{Sakaki1987}
\begin{equation}
W(x,y)=\frac{\partial {{E}_{n}}}{\partial a}\Delta (x,y)\equiv {{V}_{n}}\Delta (x,y)
\end{equation}

Например, для прямоугольной квантовой ямы (КЯ) с бесконечными стенками для потенциала:
\begin{equation}
E_n = \frac{\hbar^2 \pi^2 n^2}{2m a^2}=E_0 n^2 \Rightarrow V_n = -\frac{2}{E_0}n^2
\end{equation}

Для случая квантовой ямы с параболическим потенциалом для электрона с эффективной массой $m_e$:
\begin{equation}
E_n=2\hbar {\left[\frac{2\Delta E_c} {m_e}\right]}^{\frac{1}{2}}\frac{1}{a}\left(n+\frac{1}{2}\right) = \hbar \omega_e\left(n+\frac{1}{2}\right), V_n=-\frac{1}{a}\hbar \omega_e\left(n+\frac{1}{2}\right)
\end{equation}
$\Delta E_c$ --- высота параболического потенциала на границе наноструктуры, $\hbar \omega_e$ --- энергия размерного квантования.

Часто в расчетах кинетических коэффициентов используется случай гауссовой флуктуации поверхности, когда автокорреляционная функция для различных точек поверхности определяется соотношением:
\begin{equation}
\left\{\Delta (x,y)\Delta (x',y')\right\}_{V} =\Delta ^{2} {\rm exp}\left[-\frac{1}{\Lambda ^{2} } \left((x-x')^{2} +(y-y')^{2} \right)\right] \equiv F\left(\left|\boldsymbol{\rho} - \boldsymbol{\rho '} \right|\right)
\end{equation}