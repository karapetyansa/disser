\chapter{Рассеяния носителей на шероховатой поверхности} \label{chapt1}

\section{Обзор литературы} \label{sect1_1}

\section{Теория рассеяния носителей на шероховатой поверхности в размерно-ограниченных системах} \label{sect1_2}
Из-за неровности поверхности случайным образом меняется ширина $a$ размерно-ограниченной системы, что приводит к флуктуации энергии размерного квантования $E_n$ при движении носителя параллельно поверхности исследуемой квантовой системы. Следовательно, энергия взаимодействия электрона (дырки) с шероховатой поверхностью в случае двухмерного электронного газа может быть записана в следующем виде \cite{Sakaki1987}
\begin{equation}
\label{eq:1_1}
W(x,y)=\frac{\partial E_n}{\partial a}\Delta(x,y)\equiv V_n \Delta(x,y)
\end{equation}

Например, для прямоугольной квантовой ямы (КЯ) с бесконечными стенками для потенциала:
\begin{equation}
\label{eq:1_2}
E_n = \frac{\hbar^2 \pi^2 n^2}{2ma^2}=E_0 n^2 \Rightarrow V_n = -\frac{2}{E_0}n^2
\end{equation}

Для случая квантовой ямы с параболическим потенциалом для электрона с эффективной массой $m_e$:
\begin{equation}
\label{eq:1_3}
E_n=2\hbar \left[ \frac{2\Delta E_c} {m_e} \right]^\frac{1}{2} \frac{1}{a}\left( n + \frac{1}{2} \right) = \hbar \omega_e \left( n + \frac{1}{2} \right), V_n = -\frac{1}{a} \hbar \omega_e \left( n + \frac{1}{2} \right)
\end{equation}
$\Delta E_c$ --- высота параболического потенциала на границе наноструктуры, $\hbar \omega_e$ --- энергия размерного квантования.

Часто в расчетах кинетических коэффициентов используется случай гауссовой флуктуации поверхности, когда автокорреляционная функция для различных точек поверхности определяется соотношением:
\begin{equation}
\label{eq:1_4}
\left\{ \Delta(x,y)\Delta(x',y') \right\}_V = \Delta^2 \exp \left[ - \frac{1}{\Lambda^2} \left( (x-x')^2 +(y-y')^2 \right) \right] \equiv F \left( \left| \boldsymbol{\rho} - \boldsymbol{\rho'} \right| \right),
\end{equation}
здесь:
$\Delta, \Lambda$ --- высота и ширина гауссовой флуктуации соответственно, 
$\left\{ ... \right\}_V$
описывает усреднение по реализации случайного процесса 
$\Delta(x,y)$, $\left| \boldsymbol{\rho} - \boldsymbol{\rho'} \right| = (x-x')^2 + (y-y')^2$.
Естественно, что можно рассматривать случай $\delta$-образной флуктуации \cite{Lozovik1998}, когда
\begin{equation}
\label{eq:1_5}
\left\{ \Delta(x,y)\Delta(x',y') \right\} = \gamma\delta\left( \boldsymbol{\rho} - \boldsymbol{\rho'} \right) = \gamma_0\delta(x-x')(y-y')=\tilde{F} \left(\left| \boldsymbol{\rho} - \boldsymbol{\rho'} \right|\right)
\end{equation}
$\gamma_0$ --- определяет квадрат амплитуды флуктуации.

Если исследовать случай одномерного электронного газа (примером могут служить квантовые проволоки, квантовые нанотрубки), то для гауссовой флуктуации поверхности автокорреляционная функция для различных точек поверхности по аналогии (\ref{eq:1_4}) может быть записана следующим образом:
\begin{equation}
\label{eq:1_6}
\left\{\Delta(x)\Delta(x')\right\}=\Delta^2_0 \exp \left[-\frac{(x-x')}{\Lambda^2_0}\right] = F_0(x-x')
\end{equation}
Для случая $\delta$-образной флуктуации поверхности естественно положить:
\begin{equation}
\label{eq:1_7}
\left\{\Delta(x)\Delta(x')\right\}= \gamma\delta(x-x') = \tilde{F}_0(x-x')
\end{equation}

Время релаксации носителей на шероховатой поверхности, определяется квантово-механической вероятностью рассеяния в единицу времени, в нижайшем порядке теории возмущений определяется соотношением:
\begin{equation}
\label{eq:1_8}
\frac{1}{\tau_a}=\frac{2\pi}{\hbar} \sum_\beta{V_\alpha V_\beta W_{\alpha\beta}\delta\left(\varepsilon_\alpha-\varepsilon_\beta\right)}
\end{equation}
\[
W_{\alpha\beta}=\int{d\vect{r} d\vect{r_1} \Psi^*_\alpha(\vect{r}) \Psi^*_\beta(\vect{r_1}) V_\alpha V_\beta F \Psi_\alpha(\vect{r_1}) \Psi_\beta(\vect{r})}
\]
$\Psi_\alpha(\vect{r_1})$--- волновая функция носителей в состоянии $\alpha$ в размерно-квантованной системе. $F$ определяется соотношением (\labelcref{eq:1_4,eq:1_5,eq:1_6,eq:1_7}).

{\color{red} Вставить выражение для $W_{\alpha\beta}$, аналогично тому как ниже вставили для проволок}

Для квазидвумерных систем (квантовые ямы с различным профилем потенциала, гетероструктуры) в случае гауссовой флуктуации поверхности (\ref{eq:1_4}) не трудно получить:
\begin{equation}
\label{eq:1_9}
\frac{1}{\tau_a}=\frac{\pi m_e}{\hbar^3}{\left(\Delta \Lambda V_n\right)}^2 \exp{\left[-\frac{1}{2}(\Lambda k_\bot )^2\right]} I_0 \left[\frac{1}{2}(\Lambda k_\bot)^2\right]
\end{equation}
$I_0(z)$ – модифицированная функция Бесселя нулевого значка, $k_\bot = \sqrt{k^2_x+k^2_y}$ – волновой вектор электрона в плоскости двумерной системы.

При низких температурах, когда $\Lambda k_\bot \ll 1$ из (\ref{eq:1_9}) следует:
\begin{equation}
\label{eq:1_10}
\frac{1}{\tau_a}=\frac{\pi m_e}{\hbar^3}{\left(\Delta \Lambda V_n\right)}^2
\end{equation}
Аналогично можно записать для случая $\delta$-образной флуктуации поверхности
\begin{equation}
\label{eq:1_11}
\frac{1}{\tau_a}=\frac{m_e\gamma}{\hbar^3}V_n^2
\end{equation}
Согласно (\ref{eq:1_10}), (\ref{eq:1_11}) рассеяние электронов происходит в одной зоне, и время релаксации зависит только от номера размерно-квантованной зоны $n$. При этом с ростом $n$ $\tau_a$, если учитывать (\ref{eq:1_2}), (\ref{eq:1_3}) уменьшается.

В случае прямоугольной квантовой ямы:
\begin{equation}
\frac{1}{\tau_\alpha} \sim \frac{1}{a^2} E_0^2 n^4 
\end{equation}
А для параболической квантовой ямы:
\begin{equation}
\frac{1}{\tau_\alpha} \sim \frac{1}{a^2}(\hbar\omega_e)^2{\left(n + \frac{1}{2}\right)}^2
\end{equation}
Следовательно, для двумерного электронного газа время релаксации при рассеянии на шероховатой поверхности (\ref{eq:1_10}), (\ref{eq:1_11})  справедливо для произвольного вида потенциала $V(\vect{\rho})$:
\[
\left[-\frac{\hbar^2}{2m_e}\frac{\partial^2}{\partial \vect{\rho}^2}+V(\vect{\rho})\right]\Psi_n(\vect{\rho})=E_n\Psi_n(\vect{\rho}).
\]

Для одномерных квантовых систем (например нанопроволоки, нанотрубки) когда носители свободно движутся вдоль оси $OX$ исследуемой наноструктуры и волновая функция определяется как
\[
\psi_\alpha(\vect{r})=Ce^{ik_x}\varphi(y,z),
\]
то согласно (\ref{eq:1_8}) для гаусовой флуктуации поверхности
\begin{equation}
W_{\alpha\beta }=\frac{\Delta^2_0}{L_x}\sqrt{\pi}\Lambda_0 \exp\left[-\frac{(k_x-k'_x)\Lambda^2_0}{4}\right] \delta_{nn'},
\end{equation}
для $\delta$-образной флуктуации поверхности
\begin{equation}
W_{\alpha\beta }=\frac{\gamma_0}{L_x}\delta_{nn'}.
\end{equation}
Наличие символов Кронекера указывает, что процессы рассеяния носителей происходят в одной зоне. Следовательно согласно (\ref{eq:1_8}) время релаксации определяется соотношением
{\color{red}
\[
\frac{1}{\tau_\alpha}=\frac{2\pi}{\hbar }\frac{L_x}{2\pi }{\left|V_a\right|}^2\int{W_{\alpha\beta}\delta(k_x-k'_x)dk'_x}.
\]
}
{\color{red}Нет суммы по $\beta$ (подразумевается что там дельта символ, но зачем тогда писать $\beta$) и $2\pi$ и в числителе и знаменателе}.

Для квадратичного закона дисперсии
\[
\varepsilon(k_x)=\frac{\hbar^2 k^2_x}{2m_e} 
\] 
не трудно получить для гауссовой флуктуации
\begin{equation}
\frac{1}{\tau_\alpha}=\frac{m_e}{\hbar^3} \frac{\left|V_a\right|^2}{\left|k_x\right|} \frac{\Delta^2_0\Lambda_0}{2}\sqrt{\pi}\left(1+\exp\left[-\Lambda^2_0 k^2_x \right] \right)
\end{equation}
для $\delta$-образной флуктуации поверхности
\begin{equation}
\label{eq:1_17}
\frac{1}{\tau_\alpha}=\frac{m_e}{\hbar^3} \frac{\left|V_a\right|^2}{\left|k_x\right|}\gamma_0
\end{equation}
Следовательно время релаксации зависит от номера размерно-квантованной зоны и имеет особенности при $k_x=0$, т.е. на дне зоны проводимости. Последнее обстоятельство является непосредственным следствием одномерности движения носителей заряда. Для параболических квантовых проволок радиуса $R$ энергетический спектр зонных электронов, когда магнитное поле $\vect{B}$ направленно перпендикулярно оси нанопроволоки, а электрическое поле $\vect{E}$ параллельно $\vect{B}$, определяется аналогично \cite{Geiler1998} и имеет вид:
\begin{equation}
E_{k_x,n,m}=\frac{\hbar^2 k^2_x}{2m^*}+\hbar\Omega_e\left(n+\frac{1}{2}\right)+\hbar \omega_e\left(m+\frac{1}{2}\right)-\Delta_c
\end{equation}
Здесь обозначено
\[
m^*=m_e\left(\frac{\Omega_e}{\omega_e}\right)^2, \:
\Omega^2_e=\omega_{ec}^2+\omega_c^2, \:
\omega_{ec}=\frac{eH}{m_e c}, \:
\omega_e=\frac{1}{R}{\left[\frac{2\Delta E_c}{m_e}\right]}^\frac{1}{2}, \:
\Delta_c=\frac{\left(eER\right)^2}{4\Delta E_c}
\] 
Следовательно, согласно (\ref{eq:1_1}):
\begin{equation}
\label{eq:1_19}
V_\alpha=-\frac{1}{R} \left[\left(\frac{\omega_e\omega_{ec}}{\Omega_e}\right)^2 \frac{\hbar^2 k^2_x}{m_e}+\hbar\omega_e \left(\frac{\omega_e}{\Omega_e}\right) \left(n+\frac{1}{2}\right) + \hbar\omega_e \left(m+\frac{1}{2}\right) + 2\Delta_c \right]
\end{equation}
В дальнейшем исследуются кинетические явления при низких температурах, когда процессы рассеяния носителей на шероховатой поверхности являются наиболее активными. Но при низких температурах в процессах переноса принимают участие электроны с малыми значениями волнового вектора, поэтому зависимость $V_\alpha$ от волнового вектора можно пренебречь, если $\hbar\omega_e \gg k_0 T$. Последнее неравенство хорошо выполняется в области низких температур, когда размерно-квантованные уровни проявляются наиболее ярко. В рассматриваемых приближения время релаксации с учетом (\ref{eq:1_17}), (\ref{eq:1_19}) для случая $\delta$-образной флуктуации \footnote{Заметим, что для случая гауссовой флуктуации поверхности при $\Lambda_0 k_x<1$ нужно $\gamma_0$ заменить на $\Delta_0^2 \Lambda_0 \sqrt{\pi}$} принимает вид:
\begin{equation}
\label{eq:1_20}
\frac{1}{\tau_\alpha}=\frac{m^*_e\gamma_0}{\hbar^3 R^2} \left[\left(\frac{\omega_e}{\Omega_e}\right)\left(n+\frac{1}{2}\right) + \left(m+\frac{1}{2}\right) + \frac{2\Delta_c}{\hbar \omega_e}\right]^2 \frac{\left(\hbar \omega_e\right)^2}{\left|k_x\right|}
\end{equation}
Из (\ref{eq:1_20}) непосредственно следует, что с уменьшением размеров наноструктуры время релаксации существенно уменьшается $\left(\tau_\alpha \sim R^4 \right)$. Последнее обстоятельство позволяет экспериментально выделять рассматриваемый механизм рассеяния \cite{Sakaki1987} от других конкурирующих механизмов рассеяния при исследовании явлений переноса.

С ростом напряженности магнитного поля время релаксации уменьшается, что связано с увеличением локализации зонных носителей. Поперечное электрическое поле прижимает электроны к поверхности исследуемой наноструктуры, поэтому вероятность рассеяния носителей на шероховатой поверхности увеличивается. Именно по этой причине время релаксации уменьшается, что, естественно, должно влиять на кинетические коэффициенты (электропроводность, термоэдс). Если $\vect{B}\bot\vect{E}$ (оба вектора расположены в плоскости перпендикулярной оси квантовой проволоки), то
\begin{equation}
\frac{1}{\tau_\alpha}=\frac{2m_e\Omega^2_e \gamma_0}{\hbar R^2 \left|k_x\right|} \left[\left(\frac{\omega_e}{\Omega_e}\right) \left(n+\frac{1}{2}\right)+ \left(m+\frac{1}{2}\right)+ \frac{2\Delta_c}{\hbar\Omega_e} \left(\frac{\omega_e}{\Omega_e}\right)^3\right]^2
\end{equation}

Следовательно заметная зависимость $\tau_\alpha$ от напряженности поперечного электрического поля проявляется при более больших значениях $E$ чем в случае $\vect{B}||\vect{E}$.
Заметим, что только процессы рассеяния носителей на шероховатой поверхности (как для одномерного, так и для квазидвумерного электронного газа) зависят от напряженности постоянного поперечного электрического поля.
Согласно классической теории явлений переноса в конденсированных средах, использующей решение кинетического уравнения Больцмана, кинетические коэффициенты определяются транспортным временем релаксации, которое имеет вид \cite{Bonch1977}:
\begin{equation}
\frac{1}{\tau^{tr}_\alpha}=\frac{2\pi}{\hbar }\sum_\beta{W_{\alpha\beta}(1-\cos\theta) \delta\left(\varepsilon_\alpha-\varepsilon_\beta\right)}
\end{equation}
$\theta$ --– угол между волновыми векторами $\vect{k}$ и $\vect{k'}$, характеризующие движение электронов до и после процессов рассеяния. Для случая рассеяния носителей на гауссовой флуктуации
\begin{multline}
\label{eq:1_23}
\frac{1}{\tau^{tr}}= \frac{\Delta^2_0\Lambda_0}{\hbar^3} \left|V_n\right|^2\frac{m_e}{2} \int^{2\pi}_0{d\theta(1-\cos\theta) e^{-\frac{k^2_\bot\Lambda^2_0}{2}(1-\cos\theta)}} = \\
= \frac{\pi m_e}{\hbar^3}\left(\Delta_0\Lambda_0 V_n\right)^2 e^{-\frac{k^2_\bot\Lambda^2_0}{2}} \left[I_0\left(\frac{k^2_\bot\Lambda^2_0}{2}\right)-I_1\left(\frac{k^2_\bot\Lambda^2_0}{2}\right)\right]
\end{multline}
Заметим, что при $k^2_\bot\Lambda^2_0 \ll 1$ (\ref{eq:1_23}) совпадает с временем релаксации (\ref{eq:1_9}).

Обратим внимание, что времена релаксации для размерно-ограниченных систем (квантовые ямы, нанопроволоки), определяемые рассеянием носителей на шероховатой поверхности, для $\delta$-образной флуктуации в точности совпадают с транспортным временем релаксации.
С ростом температуры и с увеличением размеров наноструктуры (ширины квантовой ямы, радиуса нанопроволоки) на кинетические явления в исследуемых квантовых системах начинают влиять процессы рассеяния носителей на колебаниях кристаллической решетки. В случае упругого рассеяния носителей на длинноволоновых колебаниях, когда $\hbar\omega_q \ll 1$ ($\hbar\omega_q$ --- энергия акустических фононов с волновым вектором $\vect{q}$) время релаксации определяется соотношением \cite{Sinyavskii2006}:
\begin{equation}
\frac{1}{\tau^{ph}_\alpha} = \frac{2\pi E^2_1 k_0 T}{\pi\rho v^2} \sum_\beta{ \int{d\vect{r}\left|\Psi_\alpha(\vect{r})\right|^2 \left|\Psi_\beta(\vect{r})\right|^2 \delta\left(\varepsilon_\alpha-\varepsilon_\beta\right)}}
\end{equation}
$E_1$ --- константа деформационного потенциала для электронов, $v$ --- скорость звука в размерно-ограниченной системе плотность $\rho$.

Для параболической квантовой проволоки в поперечном магнитном поле время релаксации при рассеянии носителей в нижайшей размерно-квантованной зоне проводимости определяется соотношением:
\begin{equation}
\label{eq:1_25}
\frac{1}{\tau^{ph}_\alpha}=\frac{E^2_1 m^2_e \omega_e k_0 T}{\pi \rho v^2 \hbar^4}\sqrt{\frac{2}{\pi}} \frac{1}{\left|k_x\right|}
\end{equation}
Согласно (\ref{eq:1_25}) и (\ref{eq:1_20}), как показывают расчеты \cite{SinyavskiiKostyukevich2013, SinyavskiiSolovenko2014}, влиянием рассеяния носителей заряда на длинноволновых акустических колебаниях можно пренебречь, если

\begin{equation}
\label{eq:1_26}
T\left[\frac{R \cdot 10^{-2}}{\gamma^{\frac{1}{3}}_0}\right] \ll 1
\end{equation}
При записи (\ref{eq:1_26}) учитывались параметры типичные для квантовых проволок $E_1=10 \text{eV}$, $v=1.5\cdot 10^5 \text{cm/s}$, $m_e=0.06 m_0$, $\rho=5.4\text{g}/\text{cm}^3$. Следовательно при $T=4^{\circ} \text{K}$  $\gamma_0^{\frac{1}{3}} \cong 12 \AA$ можно пренебрегать взимодействием электронов с колебаниями кристаллической решетки, если $R \ll 10^3 \AA$.