\chapter{Рассеяния носителей на шероховатой поверхности} \label{chapt1}

\section{Обзор литературы} \label{sect1_1}

\section{Теория рассеяния носителей на шероховатой поверхности в размерно-ограниченных системах} \label{sect1_2}
Из-за неровности поверхности случайным образом меняется ширина $a$ размерно-ограниченной системы, что приводит к флуктуации энергии размерного квантования $E_n$ при движении носителя параллельно поверхности исследуемой квантовой системы. Следовательно, энергия взаимодействия электрона (дырки) с шероховатой поверхностью в случае двухмерного электронного газа может быть записана в следующем виде \cite{Sakaki1987}
\begin{equation}
W(x,y)=\frac{\partial E_n}{\partial a}\Delta(x,y)\equiv V_n \Delta(x,y)
\end{equation}

Например, для прямоугольной квантовой ямы (КЯ) с бесконечными стенками для потенциала:
\begin{equation}
E_n = \frac{\hbar^2 \pi^2 n^2}{2ma^2}=E_0 n^2 \Rightarrow V_n = -\frac{2}{E_0}n^2
\end{equation}

Для случая квантовой ямы с параболическим потенциалом для электрона с эффективной массой $m_e$:
\begin{equation}
E_n=2\hbar \left[ \frac{2\Delta E_c} {m_e} \right]^\frac{1}{2} \frac{1}{a}\left( n + \frac{1}{2} \right) = \hbar \omega_e \left( n + \frac{1}{2} \right), V_n = -\frac{1}{a} \hbar \omega_e \left( n + \frac{1}{2} \right)
\end{equation}
$\Delta E_c$ --- высота параболического потенциала на границе наноструктуры, $\hbar \omega_e$ --- энергия размерного квантования.

Часто в расчетах кинетических коэффициентов используется случай гауссовой флуктуации поверхности, когда автокорреляционная функция для различных точек поверхности определяется соотношением:
\begin{equation}
\label{eq:1_4}
\left\{ \Delta(x,y)\Delta(x',y') \right\}_V = \Delta^2 \exp \left[ - \frac{1}{\Lambda^2} \left( (x-x')^2 +(y-y')^2 \right) \right] \equiv F \left( \left| \boldsymbol{\rho} - \boldsymbol{\rho'} \right| \right),
\end{equation}
здесь:
$\Delta, \Lambda$ --- высота и ширина гауссовой флуктуации соответственно, 
$\left\{ ... \right\}_V$
описывает усреднение по реализации случайного процесса 
$\Delta(x,y)$, $\left| \boldsymbol{\rho} - \boldsymbol{\rho'} \right| = (x-x')^2 + (y-y')^2$.
Естественно, что можно рассматривать случай $\delta$-образной флуктуации \cite{Lozovik1998}, когда
\begin{equation}
\label{eq:1_5}
\left\{ \Delta(x,y)\Delta(x',y') \right\} = \gamma\delta\left( \boldsymbol{\rho} - \boldsymbol{\rho'} \right) = \gamma_0\delta(x-x')(y-y')=\tilde{F} \left(\left| \boldsymbol{\rho} - \boldsymbol{\rho'} \right|\right)
\end{equation}
$\gamma_0$ --- определяет квадрат амплитуды флуктуации.

Если исследовать случай одномерного электронного газа (примером могут служить квантовые проволоки, квантовые нанотрубки), то для гауссовой флуктуации поверхности автокорреляционная функция для различных точек поверхности по аналогии (\ref{eq:1_4}) может быть записана следующим образом:
\begin{equation}
\label{eq:1_6}
\left\{\Delta(x)\Delta(x')\right\}=\Delta^2_0 \exp \left[-\frac{(x-x')}{\Lambda^2_0}\right] = F_0(x-x')
\end{equation}
Для случая $\delta$-образной флуктуации поверхности естественно положить:
\begin{equation}
\label{eq:1_7}
\left\{\Delta(x)\Delta(x')\right\}= \gamma\delta(x-x') = \tilde{F}_0(x-x')
\end{equation}

Время релаксации носителей на шероховатой поверхности, определяется квантово-механической вероятностью рассеяния в единицу времени, в нижайшем порядке теории возмущений определяется соотношением:
\begin{equation}
\frac{1}{t_a}=\frac{2\pi}{\hbar} \sum_\beta{V_\alpha V_\beta W_{\alpha\beta}\delta\left(\epsilon_\alpha-\epsilon_\beta\right)}
\end{equation}
\[
W_{\alpha\beta}=\int{d\vect{r} d\vect{r_1} \Psi^*_\alpha(\vect{r}) \Psi^*_\beta(\vect{r_1}) V_\alpha V_\beta F \Psi_\alpha(\vect{r_1}) \Psi_\beta(\vect{r})}
\]
$\Psi_\alpha(\vect{r_1})$--- волновая функция носителей в состоянии $\alpha$ в размерно-квантованной системе. $F$ определяется соотношением (\labelcref{eq:1_4,eq:1_5,eq:1_6,eq:1_7}).

{\color{red} Вставить выражение для $W_{\alpha\beta}$, аналогично тому как ниже вставили для проволок}

Для квазидвумерных систем (квантовые ямы с различным профилем потенциала, гетероструктуры) в случае гауссовой флуктуации поверхности \label{eq:1_4} не трудно получить:
\begin{equation}
\frac{1}{\tau_a}=\frac{\pi m_e}{\hbar^3}{\left(\Delta \Lambda V_n\right)}^2 e^{-\frac{1}{2}{\left(\Lambda k_\bot \right)}^2} I_0 \left[\frac{1}{2}{\left(\Lambda k_\bot \right)}^2\right]
\end{equation}

