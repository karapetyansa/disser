\chapter{Влияние шероховатой поверхности на кинетические эффекты в низкоразмерных системах} \label{chapt3}

\section{Подвижность носителей в размерно-квантованных системах с учетом рассеяния на поверхности и фононах} \label{sect3_1}

Квантовые системы с пониженной размерностью (квантовые ямы (КЯ), сверхрешетки, квантовые проволоки (КП)) благодаря их уникальным свойствам, связанным с возникновением размерного квантования, продолжают привлекать внимание теоретиков и экспериментаторов. При этом кинетические явления в размерно-квантованных системах принципиальным образом отличаются от объемных материалов. 

\subsection{Постановка задачи. Общие соотношения} \label{subsect3_1_1}

\section{Электропроводность в размерно-квантовых системах с учётом рассеяния носителей на поверхности в магнитном поле} \label{sect3_2}

Рассмотрим особенности электропроводности, возникающие в размерно-квантованных системах в присутствии однородного магнитного поля напряжённостью...
