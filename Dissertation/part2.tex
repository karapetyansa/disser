\chapter{Влияние рассеяния носителей на шероховатой поверхности на оптические свойства размерно-ограниченных систем} \label{chapt2}

\section{Межзонное и внутризонное поглощение света в квантовых системах с пониженной размерностью с учетом рассеяния на шероховатой поверхности} \label{sect2_1}

В наноструктурах зона проводимости (как и валентная зона) квантуется, поэтому возникают новые каналы поглощения (излучения) слабой электромагнитной волны в далекой инфракрасной области спектра, связанные с переходом электрона между размерно-квантованными состояниями зоны проводимости...
%\newpage
%============================================================================================================================
\section{Влияние лазерного излучения на оптические свойства квантовых пленок} \label{sect2_2}

Исследования резонансных явлений в физике твердого тела являются довольно привлекательными, так как в этом случае лазерное излучение приводит к заметному влиянию на кинетические явления даже при небольших интенсивностях электромагнитной волны.

\section{Оптические свойства квантовых поволок в присутствии резонансного лазерного излучения} \label{sect2_3}

Рассмотрим квантовую проволоку в однородном магнитом поле ...
